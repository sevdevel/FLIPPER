\documentclass[10pt]{article}
\usepackage{graphicx}
\usepackage{amssymb,amsmath}
\usepackage{siunitx}
\usepackage[a4paper]{geometry}
\usepackage{amsmath}
\usepackage{multirow}
\usepackage{hyperref}
\usepackage{color}
\usepackage{tabularx}
\usepackage{multirow}
\usepackage{supertabular}

% Bibliography
\usepackage[style=alphabetic,sorting=nyt,sortcites=true,autopunct=true,babel=hyphen,hyperref=true,abbreviate=false,backref=true,backend=biber]{biblatex}
\addbibresource{FLIPPERDOC.bib} % BibTeX bibliography file
\defbibheading{bibempty}{}

% Index
\usepackage{calc} % For simpler calculation - used for spacing the index letter headings correctly
\usepackage{makeidx} % Required to make an index
\makeindex % Tells LaTeX to create the files required for indexing


\begin{document}

\begin{titlepage}

\newcommand{\HRule}{\rule{\linewidth}{0.5mm}} % Defines a new command for the horizontal lines, change thickness here

\center % Center everything on the page
 
%----------------------------------------------------------------------------------------
%	HEADING SECTIONS
%----------------------------------------------------------------------------------------

\textsc{\LARGE Team FLIPPER presents}\\[1.5cm] % Name of your university/college
\vspace{3cm}
\textsc{\Large A short manual for}\\[0.5cm] % Major heading such as course name

%----------------------------------------------------------------------------------------
%	TITLE SECTION
%----------------------------------------------------------------------------------------

\HRule \\[0.4cm]
{ \huge \bfseries FLIPPER: FLexible Interpretation of Porewater Profiles and Interpretation of Rates }\\[0.4cm] % Title of your document
\HRule \\[1.5cm]
 
%----------------------------------------------------------------------------------------
%	AUTHOR SECTION
%----------------------------------------------------------------------------------------

\begin{minipage}{0.4\textwidth}
\begin{flushleft} \large
\emph{Author:}\\
Sebastiaan \textsc{van de Velde} % Your name
\end{flushleft}
\end{minipage}
~
\begin{minipage}{0.4\textwidth}
\begin{flushright} \large
\emph{With input from:} \\
... \textsc{...} % Supervisor's Name
\end{flushright}
\end{minipage}\\[2cm]

% If you don't want a supervisor, uncomment the two lines below and remove the section above
%\Large \emph{Author:}\\
%John \textsc{Smith}\\[3cm] % Your name

%----------------------------------------------------------------------------------------
%	LOGO SECTION
%----------------------------------------------------------------------------------------

%\includegraphics{ECOMOD-CommonSetupGuide_Figures/Logo_RBINS}\\[0.5cm] % Include a department/university logo - this will require the graphicx package

%----------------------------------------------------------------------------------------
%	DATE SECTION
%----------------------------------------------------------------------------------------
\vspace{3cm}
{\large \today}\\[3cm] % Date, change the \today to a set date if you want to be precise

 
%----------------------------------------------------------------------------------------

\vfill % Fill the rest of the page with whitespace

\end{titlepage}
\pagenumbering{roman}
\setcounter{tocdepth}{3}
\tableofcontents
\pagebreak
\setcounter{page}{1}
\pagenumbering{arabic}


\section{Introduction}

Under construction

\section{Required software and packages}
\label{sect_software}

FLIPPER is essentially a series of functions and scripts written in the R language. Hence, you will need a functioning version of R and Rstudio. 

Required packages:
\begin{itemize}
%\item tcltk (basepackage)
\item marelac
\item signal
\item fractaldim
\item ReacTran
\item marelac
\item FME
\item wavelets
\end{itemize}

\section{Package functioning}
\label{sect_functioning}

The package is called using the following syntax:
\begin{verbatim}
FLIPPER.func(input,por.cte=NA,E.cte=NULL,tort.dep=1,species,method=NULL,
             env.parms=NULL,full.output=FALSE,
             discrete.parms=NULL,
             continuous.parms=NULL,
             gradient.parms=NULL)
\end{verbatim}

\subsection{Input}

\subsection{Output}

% input: data for inputting: a dataframe containing 
%                          depth x         [m]
%                           concentration C [mmol m-3]
%                optionally porosity (if depth dependent) [dimensionless]
%                optionally electric field E (if depth dependent) [V m-1]
%        optionally for continuous: 
%               advective velociy (v),
%               first derivatives of porosity (dpor_dx),
%               advective velociy (dv_dx), effective diffusion coefficient (dDs_dx)
% por.cte: a constant value for porosity -> if not defined, porosity should be included 
%          in the input dataframe
% E.cte: a constant value for the electric field (optional),      [V m-1]
%     supplied as vector: c("value E","start depth","end depth")
%                                start depth = SWI (generally 0) [m]
%                                end depth   = deepest value     [m]
% tort.dep: the estimation method for tortuosity:1 (default) = 1-2ln(por), 
%                                                2           = por^-1,
%                                                3           = por^-2,
%                                                4           = 1+3(1-por)
% species: species of interest (input as in marelac; e.g. c("O2"), c("Fe") ...)  
% method: prefered method, can be "discrete", "continuous", "gradient"
%         if NULL or "all" -> all methods are run (and a suggestion of the fit-for-purpose method is given)
% full.output: logical, TRUE gives all possible output, FALSE gives a cleaned output (see below)
% env.parms: (optional) a list containing temperature TC         [degrees Celcius]
%                                     salinity S                 [PSU] 
%                                     pressure P                 [bar]
%                                     diffusion coefficient Dmol [m2 d-1]
%                                     the charge of the ion z    [dimensionless]

% parameters for discrete (optional), supplied as list (discrete.parms, only the the parms that have to be changed from default value have to be supplied) 
%        i.end: maximum number of zones to be used in discrete
%        initial.zones: start number of zones for lumping 

% parameters for continuous (optional), supplied as list (continuous.parms, only the the parms that have to be changed from default value have to be supplied) 
%        p: order of the polynomial that is fitted (typically 2 or 3) 
%        bnd.upper: integer determining filter behaviour at the upstream (upper, left) boundary. Value = 1: no constraint on flux (default value). Value = 2: constant flux imposed. Value = 3: zero flux imposed.   
%        bnd.lower: integer determining filter behaviour at the downstream (lower or right) boundary. Value = 1: no constraint on flux (default value). Value = 2: constant flux imposed. Value = 3: zero flux imposed.   
%        n.C : window size used in filtering the concentration C, representing the number of data points to the
%           left and right of the data midpoint (hence the filter length  = 2*n.C+1)
%        n.J : window size used in filtering the first order derivative dC_dx, representing the number of data points to the
%           left and right of the data midpoint (hence the filter length  = 2*n.J+1)
%        n.R : window size used in filtering the second order derivative d2C_dx2, representing the number of data points to the
%           left and right of the data midpoint (hence the filter length  = 2*n.R+1)
%        n.uniform : logical, if TRUE then n.C, n.J, and n.R are all set uniform to max(n.C,n.J,n.R)
%        min.n : minimum value of the window size, used when scanning the optimal window size 
%        max.n : maximum value of the window size, used when scanning the optimal window size
%        optimal.window.size : either "automated" or "interactive" (default) 
%                       interactive allows user to select ideal filter size
%                       automated lets function select ideal filter size
%        keep.grapics: either TRUE or FALSE (default) - keeps windows created by automated function
%        full.output : logical, TRUE gives all output, FALSE gives selected output (without info, see fitprofile_package)
%        interpolation: which mode of interpolation of the data is used (if necessary): average (take maximum stepsize) or interpolate (take minimum stepsize)

% parameters for gradient fit function (optional) supplied as list (gradient.parms, only the the parms that have to be changed from default value have to be supplied) 
%        x.limits: depth limits of the profile (a vector of 2 -> the upper depth and the lower depth)
%=============================================================================
% OUTPUT
%=============================================================================
% a list with (if applicable):
% input:             a list with user.input: the user supplied dataframe (x, C, por, tort ...)
%                                continuous.input:  the interpolated dataframe for use in the continuous function
% env.parms:         a list with the inputted environmental parameters (default with user supplied)
% gradient.method:   the output of the gradient function
% gradient.parms:    the parameters used in the gradient function (default with user supplied)
% continuous.method:         the output of the continuous function
% continuous.parms:          the parameters used in the continuous function
% discrete.method:    the output of the discrete function
% discrete.parms:     the parameters supplied in the discrete function
%=============================================================================

%\begin{figure}[!htbp]
%	\includegraphics[width=\linewidth]{ECOMOD-CommonSetupGuide_Figures/%CS_nestedgrid.png}
%	\caption{The four nested grid domains; Continental Shelf (CoS), North Sea (NoS), Southern Bight (SoB) and Belgian Coastal Zone (BeC). Each smaller domain has a higher resolution grid.}
%	\label{fig_NestedDomain}
%\end{figure}


%\begin{table*}[!h]
%	\small
%	\caption{Release and revision history of the svn 'Common Setup'.}
%	\begin{tabular}{p{1cm} p{1cm} p{1cm} p{3cm} p{1cm} p{1cm} p{1.5cm} p{1.5cm} p{2cm}}
%		\hline
%		\textbf{Release n°} & \textbf{Date} & \textbf{Author} & \textbf{Description/purpose} & \textbf{Rev. n° CS} & \textbf{Rev. n° Coherens} & \textbf{Coherens branch} & \textbf{Coherens version} & \textbf{Validation test results} \\     
%		 \hline
%		\textbf{V1.0} & June 2017 & VD & Setting up svn & - & 1003 & Trunk & V2.10.? & N/A  \\
 %       \textbf{V1.0} & June 2017 & KB & small bugs, adding the symbolic links, adding Runlepi scripts & 19 & 1009 & Trunk & V2.10.3 & N/A  \\
%        \textbf{V2.0} & Nov 2017 & PL & New format for titles in defruns (internally documented); test case output parameters for comparing future versions with older one(s); scripts for installation and running all domains  & 34 & 1060 & Trunk & V2.11 & N/A  \\
%		\textbf{V3.0} & Feb 2019 & KB & Working CSU chain & 75 & 1194 & Trunk & V2.11.2 & 
%		CSU chain works for 2009-2013, validation in progress  \\
%		 \hline
%	\end{tabular}
%	%\belowtable{} % Table Footnotes
%	\label{table_svnhistory}
%\end{table*}


\section{References}
\begin{thebibliography}{}
	
	\bibitem[Roobaertetal.(2018)]{Roobaert2018}
	Roobaert, A., Laruelle, G.G., Landschützer, P., and Regnier P.:Uncertainty in the global oceanic CO2 uptake induced by wind forcing: quantification and spatial analysis, Biogeosciences, 2018, 15:1701-1720. doi:10.5194/bg-15-1701-2018
	
	\bibitem[Wanninkhof(1992)]{Wanninkhof1992}
	Wanninkhof, R.: Relationship between wind speed and gas exchange over the ocean, Journal of Geophysical Research, 1992, 97:7373-7382.
	
	\bibitem[Wanninkhof(2014)]{Wanninkhof2014}
	Wanninkhof, R.: Relationship between wind speed and gas exchange over the ocean revisited, Limnology and Oceanography: methods, 2014, 12:351-362. doi:10.4319/lom.2014.12.351
	
	\bibitem[Weiss(1970)]{Weiss1970}
	Weiss, R.F.: The solubility of nitrogen, oxygen and argon in water and seawater, Deep-Sea Research, 1970, 17:721-735. 
	
	\bibitem[WiesenburgandGuinasso(1979)]{Wiesenburg1979}
	Wiesenburg, D.A., and Guinasso, N.L.: Equilibrium Solubilities of Methane, Carbon Monoxide, and Hydrogen in Water and Sea Water, Journal of Chemical and Engineering Data, 1979, 24:356-360. 
		
\end{thebibliography}

\section{FAQ and useful svn/linux commands}

\end{document}
